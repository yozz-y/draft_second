\documentclass[a4paper]{article}

\usepackage[english]{babel}
\usepackage[utf8]{inputenc}
\usepackage{amsmath}
\usepackage{graphicx}
\usepackage[colorinlistoftodos]{todonotes}

\title{A comparative investigation of the C-terminal tails mobility in the \beta~I
and \beta~III isoformsof tubulin.}

\author{Laurin Y.}

\date{\today}

\begin{document}
\maketitle

\begin{abstract}
Your abstract.
\end{abstract}

\section{Introduction}

\section{Material and methods}

\subsection{Human complete tubulin modelization}

%alpha sequence id => Q71U36
%beta1 sequence id => Q9H4B7
%beta3 sequence id => Q13509

For this study, we have modeled two different tublins isotypes, the \alpha1\beta1 and the \alpha1\beta3. Each
subunit was modeled independently with the same template, a double dimer in complex with the stathmin
(PDB entry : 1SA0), chosen for the highest identity (cf table x), most recently reviewed and more complete
among all crystal structures (despite being a curved conformation). It is also the best structure of tubulin
already integrated in the model database of the Modeller program which we used for the modelization process (v9.14).
(The tubulin type model already present in the Modeller v9.14 database are 1SA0, 1TUB and FtZ like domain models, the
prokaryote counterpart of the tubulin).
Modeller process is based on the classic comparative modeling method which consists of four
sequential steps : template selection, template-target alignment, model building and finally
model evalutaion. For the template selection, Modeller use the 3D-template matching method
(pariwise alignement with known structure only, present in an associated database). The next
step consist of an alignment between the query sequence and the structure template. In our
case, the similarity belong to the best possible template alignement, so it do not need any manual
rearrangement regarding gaps. The model building depend on the global alignment quality and the overall
completeness 
%The modelisation process was done with the program Modeller v9.14 (ref modeller).
The human tubulin sequences were obtained on the UniProtKB
website and can be found with the accession number Q71U36 for the \alpha subunit, Q9H4B7 and Q13509
for the \beta1 and \beta3 subunits respectively. After the modelization of each part independently, we reassembled them
as a straight tubulin dimer with a spatial alignment on the best resolution crystal for this conformation
(PDB entry : 1JFF, the PDB we used in our previous study).

\begin{center}
  \begin{tabular}{|c|c|c|}
    \hline
     & \alpha1 (Q71U36) & \beta1 (Q9H4B7) & \beta3 (Q13509) \\
    \hline
    1SA0 & \% sim & \% sim & \% sim \\
    \hline
    1JFF &  &  &  \\
    \hline
    1TUB &  &  &  \\
    \hline
  \end{tabular}
\end{center}

\subsection{Molecular dynamics simulation}

After this step, we submitted all models previously obtained to classic molecular dynamic simulation (five
models for each isotypes modeled) in order to relax the structure and to explore the structural variability
of the C-termini newly modeled. The simulation were made with Gromacs 5.0.4 with the OPLS-AA forcefield in,
periodic boundary condition. The procedure consist of two steps of minimization, the first one is an
\textit{in vacuo} during 1000 steps without any constraints using steepest descent algorithm, followed by the addition
of a water box of 2 nm around the tubulin filled with TIP3P molecule model and neutralization of the system
with randomly added NaCl ions in the box while maintaining a 150mM concentration. The total system contain around
13 700 atoms for the tubulin dimer, 136 000 water molecules, and 300 ions (approx. 60\% Na 40\% Cl due to the high
negative charge of the C-termini).

The second minimization step is done under the same set of parameters during 5000 steps to prevent any water
clashes with the tubulin. The next stage before production phase consists of a two-steps equilibration procedure,
an NVT equilibration followed by an NPT equilibration. Each stage lasted 100 ps, with an integration step of 2 fs.
Temperature was fixed at 300 K using velocity rescale method, all bonds were constrained with the LINCS algorithm,
electrostatic interactions were computed with the Particular Mesh Ewald method. For the pressure coupling during
the NPT equilibration, we used Parinello-Rahman method at the value of 1 atm.

The production run were then performed with the same set of parameters and algorithms during 100 ns. Trajectories
were saved every 10 ps, and the analysis were made on the last 95 ns, considering the first 5 ps as equilibration
period.

% add description of the modelization process => template selection based on ? loop modeling and C-ter modeling
% side chain optimization

\subsection{Trajectory analysis}

After obtaining the simulation data, we processed them with the Gromacs 5.0.4 utilities for basics observations and
with VMD 1.9 for the graphical part. Most of the more complexes analysis were made with homemade scripts/programs.

For the clustering part, it was a two step process. Due to the high flexibility of the C-termini (mostly the one of
the \beta subunit), which are the parts of interest in this study, classicals methods can not be applied directly.
We made multiples analysis regarding the orientation, radius of gyration, contacts with the rest of the dimer to have
a preselection of subgroups. These subgroups were then submitted independently to classical hierarchical clustering
method (Gromos algorithm) in order to obtain representatives structures of differents states encountered during the
simulations. Cutoff for the clustering was 1.5 {\AA}.

%add matmet about analysis with R/python/Tcl script, clusters selection when done, RMSD/RMSF, etc.


\section{Results}

\subsection{Model building}

%table with model scores, discussion about similarity with existing model in databases, main differences b1/b3

\subsection{\beta-subunit C-terminal tail orientation and folding}

%plot theta/phi, gyration, corelation between the 2 factors, main difference b1 and b3

\subsection{Contacts \beta-subunt C-terminal with the tubulin core}

%contacts plots with differences of b1 and b3, mapping of the contacts zones

\section{Discussion}

%For the creation of the Human tubulin alpha and the two Human tubulin beta (1 and 3), we used the program
%Modeller version 9.14. The template is the same for every modeled subunit, a tubulin double dimer in complex
%with the stathmin (PDB entry : 1SA0), which was chosen due to the highest identity among tubulin crystal
%structures (cf table x) (despite being a curved conformation). The subunits were modeled independently
%and assembled after by spatial alignment in straight tubulin dimer with crystal structure of straight
%tubulin with best the resolution (PDB entry : 1JFF).
%\\
%After this first step, we made simulation for 250/350ns of each Human tubulin dimer to avoid any steric
%clash and to observe the spatial exploration of the C-termini newly modeled. Then, we made clustering on the
%complete dimer and also on the dimer without considering the highly mobile C-termini to obtain the most
%representatives structures to perform docking with the NFL-TBS.40-63 peptide.
%
%\subsection{Docking}
%
%After obtaining models of the Human tubulin, we made docking with the NFL-TBS.40-63 peptide with the same
%representatives structures that we used in our previous study (\cite{nfltbs40-63_tubulin_monpapier}. We docked
%on the representative structure of the tubulin found at the previous step without the C-termini in order to
%know if a Human tubulin will have the same behavior as crystal structure previously used (PBD entry : 1JFF).
%We also extract the best results found during this docking step to add the most commons positions of
%the C-termini on a docked tubulin for further simulations.
%
%\subsection{C-termini study and interaction}
%
%After theses steps, we took into account for the following test the C-termini in order to evaluate their
%influence and behavior. This will allow us to have a better understanding of the potentials interactions and
%influences of the C-termini on the part already studied earlier in this paper or in our previous studies.
%
%This is a test for the github editor :).
%
%This is a test from my computer ;).
%Your introduction goes here! Some examples of commonly used commands and features are listed below, to help you get started. If you have a question, please use the help menu (``?'') on the top bar to search for help or ask us a question.
%
%\section{Some \LaTeX{} Examples}
%\label{sec:examples}
%
%\subsection{How to Leave Comments}
%
%Comments can be added to the margins of the document using the \todo{Here's a comment in the margin!} todo command, as shown in the example on the right. You can also add inline comments:
%
%\todo[inline, color=green!40]{This is an inline comment.}
%
%\subsection{How to Include Figures}
%
%First you have to upload the image file (JPEG, PNG or PDF) from your computer to writeLaTeX using the upload link the project menu. Then use the includegraphics command to include it in your document. Use the figure environment and the caption command to add a number and a caption to your figure. See the code for Figure \ref{fig:frog} in this section for an example.
%
%\begin{figure}
%\centering
%\includegraphics[width=0.3\textwidth]{frog.jpg}
%\caption{\label{fig:frog}This frog was uploaded to writeLaTeX via the project menu.}
%\end{figure}
%
%\subsection{How to Make Tables}
%
%Use the table and tabular commands for basic tables --- see Table~\ref{tab:widgets}, for example.
%
%\begin{table}
%\centering
%\begin{tabular}{l|r}
%Item & Quantity \\\hline
%Widgets & 42 \\
%Gadgets & 13
%\end{tabular}
%\caption{\label{tab:widgets}An example table.}
%\end{table}
%
%\subsection{How to Write Mathematics}
%
%\LaTeX{} is great at typesetting mathematics. Let $X_1, X_2, \ldots, X_n$ be a sequence of independent and identically distributed random variables with $\text{E}[X_i] = \mu$ and $\text{Var}[X_i] = \sigma^2 < \infty$, and let
%$$S_n = \frac{X_1 + X_2 + \cdots + X_n}{n}
%      = \frac{1}{n}\sum_{i}^{n} X_i$$
%denote their mean. Then as $n$ approaches infinity, the random variables $\sqrt{n}(S_n - \mu)$ converge in distribution to a normal $\mathcal{N}(0, \sigma^2)$.
%
%\subsection{How to Make Sections and Subsections}
%
%Use section and subsection commands to organize your document. \LaTeX{} handles all the formatting and numbering automatically. Use ref and label commands for cross-references.
%
%\subsection{How to Make Lists}
%
%You can make lists with automatic numbering \dots
%
%\begin{enumerate}
%\item Like this,
%\item and like this.
%\end{enumerate}
%\dots or bullet points \dots
%\begin{itemize}
%\item Like this,
%\item and like this.
%\end{itemize}
%\dots or with words and descriptions \dots
%\begin{description}
%\item[Word] Definition
%\item[Concept] Explanation
%\item[Idea] Text
%\end{description}
%
%We hope you find write\LaTeX\ useful, and please let us know if you have any feedback using the help menu above.

\end{document}
