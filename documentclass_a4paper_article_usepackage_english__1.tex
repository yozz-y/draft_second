\documentclass[a4paper]{article}

\usepackage[english]{babel}
\usepackage[utf8]{inputenc}
\usepackage{amsmath}
\usepackage{graphicx}
\usepackage[colorinlistoftodos]{todonotes}

\title{draft papier 2}

\author{Laurin Y.}

\date{\today}

\begin{document}
\maketitle

\begin{abstract}
Your abstract.
\end{abstract}

\section{Introduction}

\section{Material and methods}

\subsection{tubulin model creation}

%alpha sequence id => Q71U36
%beta1 sequence id => Q9H4B7
%beta3 sequence id => Q13509

For the creation of the Human tubulin Alpha and the two Human tubulin Beta (1 and 3), we used the program 
Modeller version 9.14. The template is the same for every modeled subunit, a tubulin double dimer in complex 
with the stathmin (PDB entry : 1SA0), which was chosen due to the highest identity among tubulin crystal 
structures (cf table x) (despite being a curved conformation). The subunits were modeled independently 
and assembled after by spatial alignment in straight tubulin dimer with crystal structure of straight 
tubulin with best the resolution (PDB entry : 1JFF).
\\
After this first step, we made simulation for 250/350ns of each Human tubulin dimer to avoid any steric 
clash and to observe the spatial exploration of the C-termini newly modeled. Then, we made clustering on the 
complete dimer and also on the dimer without considering the highly mobile C-termini to obtain the most 
representatives structures to perform docking with the NFL-TBS.40-63 peptide.

\subsection{Docking}

After obtaining models of the Human tubulin, we made docking with the NFL-TBS.40-63 peptide with the same 
representatives structures that we used in our previous study (\cite{nfltbs40-63_tubulin_monpapier}. We docked 
on the representative structure of the tubulin found at the previous step without the C-termini in order to 
know if a Human tubulin will have the same behavior as crystal structure previously used (PBD entry : 1JFF). 
We also extract the best results found during this docking step to add the most commons positions of 
the C-termini on a docked tubulin for further simulations. 


%test from github editor

%Your introduction goes here! Some examples of commonly used commands and features are listed below, to help you get started. If you have a question, please use the help menu (``?'') on the top bar to search for help or ask us a question.
%
%\section{Some \LaTeX{} Examples}
%\label{sec:examples}
%
%\subsection{How to Leave Comments}
%
%Comments can be added to the margins of the document using the \todo{Here's a comment in the margin!} todo command, as shown in the example on the right. You can also add inline comments:
%
%\todo[inline, color=green!40]{This is an inline comment.}
%
%\subsection{How to Include Figures}
%
%First you have to upload the image file (JPEG, PNG or PDF) from your computer to writeLaTeX using the upload link the project menu. Then use the includegraphics command to include it in your document. Use the figure environment and the caption command to add a number and a caption to your figure. See the code for Figure \ref{fig:frog} in this section for an example.
%
%\begin{figure}
%\centering
%\includegraphics[width=0.3\textwidth]{frog.jpg}
%\caption{\label{fig:frog}This frog was uploaded to writeLaTeX via the project menu.}
%\end{figure}
%
%\subsection{How to Make Tables}
%
%Use the table and tabular commands for basic tables --- see Table~\ref{tab:widgets}, for example.
%
%\begin{table}
%\centering
%\begin{tabular}{l|r}
%Item & Quantity \\\hline
%Widgets & 42 \\
%Gadgets & 13
%\end{tabular}
%\caption{\label{tab:widgets}An example table.}
%\end{table}
%
%\subsection{How to Write Mathematics}
%
%\LaTeX{} is great at typesetting mathematics. Let $X_1, X_2, \ldots, X_n$ be a sequence of independent and identically distributed random variables with $\text{E}[X_i] = \mu$ and $\text{Var}[X_i] = \sigma^2 < \infty$, and let
%$$S_n = \frac{X_1 + X_2 + \cdots + X_n}{n}
%      = \frac{1}{n}\sum_{i}^{n} X_i$$
%denote their mean. Then as $n$ approaches infinity, the random variables $\sqrt{n}(S_n - \mu)$ converge in distribution to a normal $\mathcal{N}(0, \sigma^2)$.
%
%\subsection{How to Make Sections and Subsections}
%
%Use section and subsection commands to organize your document. \LaTeX{} handles all the formatting and numbering automatically. Use ref and label commands for cross-references.
%
%\subsection{How to Make Lists}
%
%You can make lists with automatic numbering \dots
%
%\begin{enumerate}
%\item Like this,
%\item and like this.
%\end{enumerate}
%\dots or bullet points \dots
%\begin{itemize}
%\item Like this,
%\item and like this.
%\end{itemize}
%\dots or with words and descriptions \dots
%\begin{description}
%\item[Word] Definition
%\item[Concept] Explanation
%\item[Idea] Text
%\end{description}
%
%We hope you find write\LaTeX\ useful, and please let us know if you have any feedback using the help menu above.

\end{document}
